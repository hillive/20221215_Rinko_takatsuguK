% twocolumn を使うと2段組になる

%\documentclass[a4j,twocolumn]{jsarticle}        % -> platex
%\documentclass[a4j,twocolumn]{ujarticle}       % -> uplatex
\documentclass[uplatex]{jsarticle}   % -> uplatex + jsarticle

\usepackage{resume} % 他パッケージ,専用コマンド,余白の設定が書かれている

%%%%%%%%%%%%%%%%%%%%%%%%%%%%%%%%%%%%%%%%%%%%%%%%%%%%%%%%%%%%%%%%%%%%%%%%
% ヘッダ: イベント名,日付,所属,タイトル,氏名
%%%%%%%%%%%%%%%%%%%%%%%%%%%%%%%%%%%%%%%%%%%%%%%%%%%%%%%%%%%%%%%%%%%%%%%%

\pagestyle{plain}
\newcommand{\comment}[1]{}
\begin{document}
\twocolumn[
\beginheader{令和n年度 コンピュータサイエンス学部 ○○発表}{2022}{12}{15}{井上 研究室}
\title{モーションベースによるによる歩行感覚の提示によるユーザーへの影響}
\author{C0B20046 川東 隆継 (Kawahigashi Takatsugu) }
\endheader
]

\vspace{3mm}

 % 本番用ページ番号オフセット
\setcounter{page}{x}

%---------------------------------------------------------------------------
% 本文
%---------------------------------------------------------------------------


\section{はじめに}
現実世界においておこうは人間の活動に欠かせない移動手段の一つであり、VRでもコンテンツ
の種類を問わず、多く登場する。従来ではロコモーションインターフェイスのフットパッド型や、
トレッドミル型などの大掛かりな装置を用いることが多かったが近年では身体的負荷の低さや
姿勢の汎用性にメリットがある座位姿勢で利用するものも提案されているが立位姿勢と比較して
歩行感覚が生まれにくいという指摘もある。そこで本研究では、座位姿勢での仮想歩行体験について、
そのユーザ体験を総合的に評価した。特に、VRコンテンツへの適用を目的とし、歩行振動、
足ふみ操作、及び仮想の歩行者の存在がユーザー体験に与える影響を検証した。

\section{提案方法}
\subsection{実験装置・環境}
本実験ではエアコンプレッサ式のモーションベースを使用し、VR視聴用のHMDにはアイトラッキング
機能を有するVive Pro Eyeを用いた。
本実験はモーションベースが稼働するのに十分なスペースを確保できる環境で行い、
実験者はモーションベースの可動域外にて実験刺激を操作した。

\section{実験}
\subsection{概要}
実験条件は以下の\tabref{tab:fonts}に示すように歩行時振動、足踏み操作、
仮想の歩行者の有無が異なる条件C1から条件C4までの4条件であった。
なお、順序効果の影響を排除するため、体験する条件の順番は参加者によりランダマイズされた。

\begin{table}
    \centering
    % \scriptsize
    \caption{フォントの変更に用いるコマンド}
    \label{tab:fonts}
    \begin{tabular}[t]{|l|l|l|l|l|}
      \hline
       & 条件 & 条件 & 条件 & 条件\\
       & C1 & C2 & C3 &C4\\
      \hline
      歩行時振動 &なし&あり&あり&あり\\
      \hline
      足ふみ操作&なし&なし&あり&あり\\ 
      \hline
      仮想の歩行者&いない&いない&いない &いる\\
      \hline
    \end{tabular}
  \end{table}
%---------------------------------------------------------------------------
% 本文終わり
%---------------------------------------------------------------------------

 % 参考文献
\bibliographystyle{junsrt}
\bibliography{ref}


\end{document}


%-----------------------------------------------------
% テンプレート
%------------------------------------------------------------------------------

%-----------
%% 箇条書き
%-----------
%\begin{itemize}
% \item
%\end{itemize}

%-------------------
%% 番号付き箇条書き
%-------------------
%\begin{enumerate}
% \item
%\end{enumerate}

%-----------
%% 図の表示
%-----------
%\begin{figure}[H]
% \centering
% \includegraphics[clip,width=7cm]{hoge.eps}
% \caption{図タイトル}\label{fig:hoge}
%\end{figure}

%-----------
%% 図の参照
%-----------
%\figref{fig:hoge}

%-----------
%% 表の作成
%-----------
%\begin{table}[H]
% \centering
% \caption{表タイトル}\label{tab:fuga}
% \begin{tabular}{|c|c|c|}\hline
%  hemo & piyo & fuga \\ \hline
%  hemo & piyo & fuga \\ \hline
% \end{tabular}
%\end{table}

%-----------
%% 表の参照
%-----------
%\tabref{tab:fuga}

%-----------
%% 参考文献
%-----------
%\begin{thebibliography}{9}
% \bibitem{piyo} 参考文献
%\end{thebibliography}

%-----------------
%% 参考文献の参照
%-----------------
%\cite{piyo}
