% twocolumn を使うと2段組になる

%\documentclass[a4j,twocolumn]{jsarticle}        % -> platex
%\documentclass[a4j,twocolumn]{ujarticle}       % -> uplatex
\documentclass[uplatex]{jsarticle}   % -> uplatex + jsarticle

\usepackage{resume} % 他パッケージ,専用コマンド,余白の設定が書かれている

%%%%%%%%%%%%%%%%%%%%%%%%%%%%%%%%%%%%%%%%%%%%%%%%%%%%%%%%%%%%%%%%%%%%%%%%
% ヘッダ: イベント名,日付,所属,タイトル,氏名
%%%%%%%%%%%%%%%%%%%%%%%%%%%%%%%%%%%%%%%%%%%%%%%%%%%%%%%%%%%%%%%%%%%%%%%%

\pagestyle{plain}
\newcommand{\comment}[1]{}
\begin{document}
\twocolumn[
\beginheader{令和n年度 コンピュータサイエンス学部 ○○発表}{2022}{12}{15}{井上 研究室}
\title{モーションベースによるによる歩行感覚の提示によるユーザーへの影響}
\author{C0B20046 川東 隆継 (Kawahigashi Takatsugu) }
\endheader
]

\vspace{3mm}

 % 本番用ページ番号オフセット
\setcounter{page}{x}

%---------------------------------------------------------------------------
% 本文
%---------------------------------------------------------------------------


\section{はじめに}
ゲームは1日1時間とか無理


%---------------------------------------------------------------------------
% 本文終わり
%---------------------------------------------------------------------------

 % 参考文献
\bibliographystyle{junsrt}
\bibliography{ref}


\end{document}


%-----------------------------------------------------
% テンプレート
%------------------------------------------------------------------------------

%-----------
%% 箇条書き
%-----------
%\begin{itemize}
% \item
%\end{itemize}

%-------------------
%% 番号付き箇条書き
%-------------------
%\begin{enumerate}
% \item
%\end{enumerate}

%-----------
%% 図の表示
%-----------
%\begin{figure}[H]
% \centering
% \includegraphics[clip,width=7cm]{hoge.eps}
% \caption{図タイトル}\label{fig:hoge}
%\end{figure}

%-----------
%% 図の参照
%-----------
%\figref{fig:hoge}

%-----------
%% 表の作成
%-----------
%\begin{table}[H]
% \centering
% \caption{表タイトル}\label{tab:fuga}
% \begin{tabular}{|c|c|c|}\hline
%  hemo & piyo & fuga \\ \hline
%  hemo & piyo & fuga \\ \hline
% \end{tabular}
%\end{table}

%-----------
%% 表の参照
%-----------
%\tabref{tab:fuga}

%-----------
%% 参考文献
%-----------
%\begin{thebibliography}{9}
% \bibitem{piyo} 参考文献
%\end{thebibliography}

%-----------------
%% 参考文献の参照
%-----------------
%\cite{piyo}
